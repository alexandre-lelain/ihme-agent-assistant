\chapter{Spécifications}

Dans cette partie, nous décrirons le principe de fonctionnement souhaité de l'agent \texttt{Chronos}, puis nous aborderons les problématiques que soulève la conception d'un tel agent.

La description de toutes ces spécifications orientera par la suite les approches théoriques et techniques retenues.


\section{Principe de l'agent souhaité}
Le principe de l'agent \texttt{Chronos} souhaité est assez simple. Il s'agit de réaliser un agent conversationnel, accessible depuis une application mobile (disponible sur \texttt{iOS} et \texttt{Android}) capable de répondre à certaines requêtes de l'utilisateur. 

Il doit être possible de dialoguer avec l'agent de façon écrite ou bien de façon orale. 

Le système de dialogue souhaité doit reposer sur un modèle de communication \og client \--- serveur \fg.

\subsection{Création d'une alarme}
L'idée principale est qu'un utilisateur doit pouvoir demander à l'agent de lui programmer une alarme. 

Pour cela, il doit lui fournir un certain nombre d'informations :
\begin{itemize}
    \item s'il veut ou non programmer une alarme ;
    \item le jour auquel programmer l'alarme ;
    \item l'heure à laquelle programmer l'alarme.
\end{itemize}

~\\\indent
Dès lors qu'il ne dispose pas de toutes ces informations, l'agent doit orienter ses questions de sorte à obtenir toutes les réponses dont il a besoin.

~\\\indent 
L'ordre dans lequel l'agent recueille ces différentes informations ne doit pas avoir d'importance. Ainsi, il doit être possible de donner l'heure avant le jour ou réciproquement.

\subsection{Alarme intelligente}
\`A partir du moment où l'agent est en possession de toutes les informations nécessaires à la programmation d'une alarme, il doit alors être capable d'obtenir la géolocalisation de l'utilisateur (couple $\{\text{longitude} / \text{latitude} \}$). 

~\\\indent 
En fonction de cette géolocalisation, l'agent doit pouvoir obtenir la météo du lieu où se situe l'utilisateur. Selon cette météo, la musique associée à l'alarme variera. 

Par exemple, s'il pleut dans la ville où est situé l'utilisateur, la musique associée à l'alarme sera une musique de pluie.

~\\\indent
La géolocalisation et la météo seront mesurées 2 heures avant le déclenchement effectif de l'alarme, de sorte à pouvoir avoir une musique la plus adaptée possible à la météo du lieu où est géolocalisé l'utilisateur.

Dans le cas où il n'est pas possible de mesurer la météo et/ou la géolocalisation 2 heures avant le déclenchement de l'alarme, les valeurs de ces deux informations correspondront à celles mesurées lors de la programmation de l'alarme.




\section{Problématiques}
L'objectif général de cet agent est qu'il soit possible de communiquer avec lui en langage naturel, c'est-à-dire dans le langage utilisé par les humains, de façon orale et écrite. 

~\\\indent
Comme nous l'avons vu plus tôt, le système de dialogue mis en place doit permettre de réunir un ensemble d'informations.

~\\\indent
L'agent doit également être robuste. Il doit en effet être capable d'indiquer à l'utilisateur s'il n'est pas capable de traiter une requête sans pour autant se mettre à disfonctionner.

~\\\indent 
Il est également souhaité que les réponses de l'agent soient rapides voire instantanées, de sorte à donner une impression de dialogue avec un autre humain. 

Les comportements de l'agents doivent donc s'apparenter le plus possible à ceux que pourraient avoir des êtres humains.

~\\\indent
Il n'est en revanche pas imposé que l'agent virtuel soit représenté par un avatar quelconque. Le but ici est de pouvoir dialoguer avec \texttt{Chronos} de façon textuelle ou orale.

~\\\indent 
Enfin, la mise en place de cet agent doit être rapide, dans le cadre d'un projet de type \emph{Proof Of Concept} (\emph{POC}).


