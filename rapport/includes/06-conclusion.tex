\chapter*{Conclusion}\addcontentsline{toc}{chapter}{Conclusion}
\thispagestyle{plain}
~\\[1cm]
{ \large

\section{Bilan}
Le bilan de se projet est satisfaisant nous avons répondus aux spécifications et à la problématique. En effet notre agent conversationnel sur mobile répond aux critères suivants établis dans les spécifications:
\begin{itemize}
	\item Il peut dialoguer en langage naturelle avec l'utilisateur;
	\item La communication peut être textuelle ou vocale;
	\item Il permet la création d'un réveil;
	\item L'ordre des informations données par l'utilisateur nécessaires à la mise en place du réveil n'a pas d'importance;
	\item Notre application récupère la géolocalisation et la météo associé de l'utilisateur pour programmer une sonnerie relative à cette météo (un son de pluie s'il pleut);
	\item L'agent est robuste et peut gérer une requête mal comprise ou n'ayant pas de rapport avec la programmation d'un réveil (\og\emph{je n'ai pas compris}\fg,\og\emph{pouvez-vous répéter ?}\fg ).
\end{itemize}

\section{Améliorations possibles}
Néanmoins ce prototype reste largement perfectible, et de nouvelles fonctionnalités pourraient être intéressante à développer. Nous avons donc lister certaines d'entre elle que nous avons retenues tout au long du projet:
\begin{itemize}
	\item Pour le moment notre application n'accède pas à la fonction réveil de Androïd, mais créer une notification avec une musique associée. Une amélioration possible serait donc de pouvoir passer directement par l'application native d'Androïd;
	\item Une autre amélioration possible serait que l'agent puisse envoyer des notifications à l'utilisateurs pour lui rappeler de programmer un réveil le soir par exemple si aucun n'est programmé. Ou d'apprendre les habitudes de l'utilisateur pour lui proposer un réveil;
	\item On peut aussi imaginer l'assistant demander si l'utilisateur préfère mettre une musique de son choix comme réveil en accédant aux musiques enregistrées sur son téléphone.
\end{itemize}
}
