\chapter*{Introduction}\addcontentsline{toc}{chapter}{Introduction}
\thispagestyle{plain}
~\\[1cm]
{ \large

Les agents assistants prennent de plus en plus de place dans notre quotidien.  Ils permettent de simplifier certaines tâches à l'utilisateur. En quelques  commandes, il est en effet aujourd'hui possible d'envoyer un message à un contact, d'obtenir des résultats sportifs et bien d'autres choses.

~\\\indent
Dans le cadre du cours d'\textsf{Intéraction Homme-Machine \'Evoluées} (\textsf{IHME}), nous avons tout d'abord effectué des travaux de recherches bibliographiques sur les systèmes de dialogues et sur les méthodes de détection d'activité. Ce premier  travail nous a permi d'obtenir les connaissances de bases nécessaires pour envisager la création de notre propre agent assistant.

~\\\indent
Une longue phase de recherche a alors débuté, de sorte à trouver des fonctionnalités à déleguer à notre futur agent qui ne sont pas encore proposées par les agents virtuels les plus répandus sur le marché (\og \texttt{Siri} \fg{} d'Apple, \og \texttt{Cortana} \fg{} de Microsoft, \og \texttt{Google Assistant} \fg{} de Google, etc.).

Plusieurs idées ont alors vu le jour, comme la création d'un agent détectant le prochain métro disponible en fonction de l'activité habituelle d'un utilisateur, d'un agent détectant les horaires habituelles d'un utilisateur pour lui indiquer à quelle heure il mangera chaque jour, d'un agent suggérant l'utilisation de l'application favorite de l'utilisateur dès lors que ce dernier a une pause dans son agenda, etc.

~\\\indent
C'est alors que nous avons eu l'idée de notre agent \texttt{Chronos}, permettant de programmer des alarmes de façon intelligente. L'idée générale est de développer un agent pour qu'un utilisateur puisse lui demander de programmer une alarme, et que la musique associée à l'alarme soit relative à la météo du lieu où se situe l'utilisateur.

~\\\indent
Nous vous expliquerons dans un premier temps le principe de notre agent ainsi que les problématiques à résoudre. Dans une seconde partie, nous traiterons les différentes solutions qui étaient possibles pour répondre à nos problématiques et celles que nous avons retenues. Par la suite sera abordée l'étape de conception de notre agent \texttt{Chronos}. Ensuite, nous vous présenterons l'implémentation de notre application et les résultats que nous avons obtenu. Nous conclurons sur les améliorations possibles de l'agent et sur ce que ce projet nous aura apporté.


}
