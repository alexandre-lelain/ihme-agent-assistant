\chapter{Installation de l'application mobile}

Vous avez la possibilité d'installer le fichier \texttt{.apk} de l'application directement sur votre \emph{SmartPhone} Android. 

Ce dernier se situe au chemin \texttt{mobile/chronos.apk}.

~\\\indent
Cette partie du manuel présente comment vous pouvez régénerer cette version compilée (pour la modifier ou l'exporter sur d'autres systèmes d'exploitation...).

\section{Pré-requis}
L'application mobile est une application écrite avec \emph{react-native}. Vous devez donc avoir installé les paquets suivants :
\begin{itemize}
\item \textbf{node} (node 9.x recommandé)
\item \textbf{npm} (npm 4.6.1 recommandé)
\item \textbf{react-native} (react-native 0.49.x recommandé)
\item \textbf{android-sdk} (dernière version recommandée)\\
\end{itemize}


L'application a également besoin d'une clef d'API \og \emph{OpenWeatherMap\_APIKey} \fg{} : elle devrait être fournie également avec le livrable\footnote{Fichier \texttt{docs/installation/accounts\_and\_api\_keys.md}}. 

Sinon, contacter un membre de l'équipe pour l'obtenir.

\section{Installation}

Se placer dans \textbf{ihme-agent-assistant/mobile/agent-assistant/}

\subsection{Dépendances}

Installation des dépendances :
\begin{lstlisting}[language=bash]
  npm install
\end{lstlisting}

\subsection{Paramétrage}

\begin{enumerate}
  \item Créer un fichier \textbf{android/local.properties} contenant :
  \begin{lstlisting}[language=bash]
  sdk.dir = [chemin vers votre sdk android]
  \end{lstlisting}
  \item Créer un fichier \textbf{components/config/config.js} contenant :
  \begin{lstlisting}[language=Javascript]
  exports.apiKey = [OpenWeatherMap_APIKey];
  \end{lstlisting}

  où \og \emph{OpenWeatherMap\_APIKey} \fg{} correspond à la clef liée à notre compte \emph{OpenWeatherMap}\footnote{Fichier \texttt{docs/installation/accounts\_and\_api\_keys.md}}.

  \item L'application mobile doit se connecter à la partie serveur pour fonctionner. Pour cela il faut indiquer l'adresse de ce dernier en
  éditant la 1ère ligne du fichier \textbf{components/api/AgentAPI.js}. Par défaut, l'url pointe vers \url{http://asi-17-ihme.insa-rouen.fr} qui est le serveur prété par l'INSA sur lequel tourne une instance de la partie serveur.
\end{enumerate}

\subsection{Lancement}

Pour lancer l'application sur votre mobile :
\begin{enumerate}
  \item Vérifiez que votre mobile est en mode débuggage
  \item Connectez votre mobile à l'ordinateur via le câble USB
  \item Compilez et lancez l'application sur votre mobile Android :
  \begin{lstlisting}[language=bash]
    react-native run-android
  \end{lstlisting}
  ou pour IOS :
  \begin{lstlisting}[language=bash]
    react-native run-ios
  \end{lstlisting}

  \textbf{Remarque : } Si vous travailler sur MacOS, le déploiement de l'application se fait par la commande :
  \begin{lstlisting}[language=bash]
    npm run my-android
  \end{lstlisting}

  \item Lancer \emph{Packager} en mode développement (optionnel):
  \begin{lstlisting}[language=bash]
    npm start
  \end{lstlisting}
\end{enumerate}

Voilà, l'application devrait maintenant être installée sur votre mobile, et votre envrionnement est configuré pour son développement.
