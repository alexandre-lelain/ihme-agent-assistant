\chapter{Installation de l'application mobile}

\section{Pré-requis}
L'application mobile est une application écrite avec \emph{react-native}. Vous devez donc avoir installé les paquets suivants :
\begin{itemize}
\item \textbf{node} (node 9.x recommandé)
\item \textbf{npm} (npm 4.6.1 recommandé)
\item \textbf{react-native} (react-native 0.49.x recommandé)
\item \textbf{android-sdk} (dernière version recommandé)\\
\end{itemize}


L'application à également besoin d'une clef d'API "OpenWeatherMap\_APIKey" : elle devrait être fournie avec le livrable. Sinon, contacter \emph{alexandre.le\_lain@insa-rouen.fr} pour l'obtenir.
\section{Installation}

Se placer dans \textbf{ihme-agent-assistant/mobile/agent-assistant/}

\subsection{Dépendances}

Installation des dépendances :
\begin{lstlisting}[language=bash]
  npm install
\end{lstlisting}

\subsection{Paramétrage}

\begin{enumerate}
  \item Créer un fichier \textbf{android/local.properties} contenant :
  \begin{lstlisting}[language=bash]
  sdk.dir = [chemin vers votre sdk android]
  \end{lstlisting}
  \item Créer un fichier \textbf{components/config/config.js} contenant :
  \begin{lstlisting}[language=Javascript]
  exports.apiKey = [OpenWeatherMap_APIKey];
  \end{lstlisting}
  \item L'application mobile doit se connecter à la partie serveur pour fonctionner. Pour cela il faut indiquer l'adresse de ce dernier en
  éditant la 1ère ligne du fichier \textbf{components/api/AgentAPI.js}. Par défaut, l'url pointe vers \emph{http://asi-17-ihme.insa-rouen.fr} qui est le serveur prété
  par l'insa sur lequel tourne une instance de la partie serveur.
\end{enumerate}

\subsection{Lancement}

Pour lancer l'application sur votre mobile :
\begin{enumerate}
  \item Vérifiez que votre mobile est en mode débuggage
  \item Connectez votre mobile à l'ordinateur via le câble USB
  \item Compiler et lancer l'application sur votre mobile Android :
  \begin{lstlisting}[language=bash]
    react-native run-android
  \end{lstlisting}
  ou pour IOS :
  \begin{lstlisting}[language=bash]
    react-native run-ios
  \end{lstlisting}
  \item Lancer Packager en mode développement (optionnel):
  \begin{lstlisting}[language=bash]
    npm start
  \end{lstlisting}
\end{enumerate}

Voilà, l'application devrait maintenant être installée sur votre mobile, et votre envrionnement est configuré pour son développement.
