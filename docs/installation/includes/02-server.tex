\chapter{Installation du Serveur}

\section{Pré-requis}
La partie serveur est un serveur \textbf{REST} dans un environnement \emph{Node.js} qui communique avec \emph{DialogFlow}, l'agent conversationnel de l'application. Vous devez donc avoir installé les paquets suivants :
\begin{itemize}
\item \textbf{node} (dernière version recommandé)
\item \textbf{npm} (dernière version recommandé)\\
\end{itemize}


L'application à également besoin d'une clef d'API "bearer\_token" qui permet d'utiliser l'API de \emph{DialogFlow}: elle devrait être fournie avec le livrable. Sinon, contacter \emph{alexandre.le\_lain@insa-rouen.fr} pour l'obtenir.
Vous devez également posséder le login et le mot de passe du compte \emph{Google} qui permet d'accéder à l'interface de \emph{DialogFlow}.

\section{Installation}

Se placer dans \textbf{ihme-agent-assistant/server/}

\subsection{Dépendances}

Installation des dépendances :
  \begin{lstlisting}[language=bash]
      npm install
  \end{lstlisting}

\subsection{Paramétrage}

Créer un fichier \textbf{src/conf.js} contenant :
\begin{lstlisting}[language=bash]
  exports.token = [bearer_token];
\end{lstlisting}

\subsection{Lancement}

Pour lancer une instance du serveur :
\begin{lstlisting}[language=bash]
  npm start
\end{lstlisting}

\subsection{DialogFlow}

Pour configurer et concevoir l'agent conversationnel (l'entité responsable du traitement et de l'analyse des dialogues) il faut se connecter à \emph{https://dialogflow.com/} avec le compte
qui vous a été fourni avec le livrable. Cette interface vous permettra de personnaliser l'IA en fonction de vos besoins.
